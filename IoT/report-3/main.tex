\documentclass[12pt,a4paper]{article}

\usepackage[fleqn]{amsmath} % This package with the fleqn option aligns equations to the left
\setlength{\mathindent}{0pt} % Set indentation from the left margin

\usepackage{amssymb} % Required for math symbols
\usepackage{graphicx} % Required for inserting images
\usepackage{geometry}

\usepackage[backend=biber, style=authoryear, citestyle=authoryear]{biblatex}
\addbibresource{references.bib}

\geometry{a4paper, margin=1in}

{
\title{
    \includegraphics[width=0.4\textwidth]{/Users/nicolasxx/documents/images/tsukuba-logo.png} \\
    \textbf{Practical Development for IoT and Embedded Systems} \\
    \vspace{3mm}    
    Report 3 on Personal Approach to Team Building
}

\author{Mamanchuk Mykola, SID.202420671}
\date{\today}
}

\usepackage{listings}
\usepackage{color}

\definecolor{codegreen}{rgb}{0,0.6,0}
\definecolor{codegray}{rgb}{0.5,0.5,0.5}
\definecolor{codepurple}{rgb}{0.58,0,0.82}
\definecolor{backcolour}{rgb}{0.99,0.99,0.99}

\lstdefinestyle{mystyle}{
    backgroundcolor=\color{backcolour},   
    commentstyle=\color{codegreen},
    keywordstyle=\color{magenta},
    numberstyle=\tiny\color{codegray},
    stringstyle=\color{codepurple},
    basicstyle=\ttfamily\footnotesize,
    breakatwhitespace=false,         
    breaklines=true,                 
    captionpos=b,                    
    keepspaces=true,                 
    numbers=left,                    
    numbersep=5pt,                  
    showspaces=false,                
    showstringspaces=false,
    showtabs=false,                  
    tabsize=2
}
\lstset{style=mystyle}

\begin{document}

\maketitle

\section{Introduction}
In the IoT class project, where we developed a touchscreen simulation using Raspberry Pi and ultrasonic sensors, I recognized the pivotal role of proactive team building for project success. Drawing from the experiences and lectures on team dynamics, \textbf{I would focus on several personal initiatives to enhance our team's effectiveness}.

\section{Considered Approaches}

Firstly, I would concentrate on facilitating \textbf{open communication}. Given the technical complexity of our project, ensuring that every team member feels comfortable voicing opinions and sharing ideas is vital. I would initiate \textbf{regular informal meetings}, where aside from project updates, team members could express any concerns or suggestions without the formal structure of our regular sessions. These could be \textbf{short, focused discussions} aimed at fostering a supportive atmosphere. \\

Secondly, I would take the initiative to \textbf{create a structured feedback system}. By implementing regular feedback loops where team members can critique and compliment one another's contributions anonymously, it would promote a culture of continuous improvement and personal accountability. This feedback would be discussed collectively, allowing us to \textbf{adjust our workflows and interactions} to better suit our project needs. \\

Furthermore, \textbf{conflict resolution} would be another area where I would take a proactive stance. By \textbf{establishing clear protocols for addressing disagreements or frustrations}, we could \textbf{prevent potential conflicts from escalating} and affecting team morale and project progress. \textbf{My role would involve being an impartial mediator during disputes}, ensuring all sides are heard and facilitating a constructive resolution. \\

Finally, to keep the team motivated and aligned with the project goals, I would organize \textbf{brief weekly/regular recap sessions}. Here, we would review our progress against the project milestones and celebrate small successes. This would not only keep everyone informed but also help \textbf{maintain enthusiasm and collective drive} towards our objectives.

\section{Conclusion}
Through these initiatives, I aim to strengthen our team's cohesion and enhance our collaborative efforts, ensuring that we not only meet but exceed our project objectives throughout the efficient workflow.

\section*{References}
\begin{enumerate}
    \item \textbf{Mamanchuk N., University of Tsukuba}, Github, \today. Available online: \url{https://github.com/RIFLE}
\end{enumerate}

\end{document}
